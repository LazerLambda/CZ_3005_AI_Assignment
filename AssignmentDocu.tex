\documentclass{article}

\title{Assignment Artificial Intelligence CZ3005 \\ Subway sandwich interactor}
\date{2019 \\ November}
\author{Koch Philipp Frederik Edward, N1903454H}

% settings
\linespread{1.5}

% imports
\usepackage{listings}
\usepackage{color}


\definecolor{orange}{rgb}{0.9531,0.39, 0.14}
\definecolor{gray}{rgb}{0.5,0.5,0.5}
\definecolor{green}{rgb}{0,0.6,0}

\lstset{
	basicstyle=\tiny,
	language=Prolog,
	numbers=left,
	numberstyle=\tiny\color{gray},
	keywordstyle=\color{blue},
	commentstyle=\color{orange},
	stringstyle=\color{green},
	lineskip={-1.5pt},
	breaklines=true,
	breakatwhitespace=true,
	frame=single
}


\begin{document}
	\maketitle
	\section{Task}
	The goal of the task was the implementation of a subway sandwich interactor. This was to guide the customer through the selection. It had to be taken into account that some options could be chosen more often, such as the chosen vegetables. Furthermore, a restriction must be made by previously selected options, so that no meat-containing ingredients are permitted in the vegetarian menu for example. Hints were offered for the task, which were also used. An \textit{options/1} rule and a \textit{selected/2} rule were proposed in the notes. The \textit{options/1} rule should offer the possible options for the respective sandwich part and the \textit{selected/2} rule should assign an option for the respective sandwich part. \textit{selected(0)} should trigger a jump on the list of sandwich parts and initiate the next assignment. If X = 1, a \textit{done/1} rule should display the options already selected. 

	\section{Implementation}
	The hints described above are very suitable for a command line based dialog program and were therefore chosen as the basis for the following implementation. For this type of program, where a rule is called again each time, an abstracted state is indispensable. The selected properties of the sandwich, the state of the selection and the number of possible reusable options must be considered. Therefore, three predicates were chosen, which should manage these states during runtime (\textit{collection/1}, \textit{state/1} and  \textit{counter/1}). Both predicates \textit{state/1} and \textit{counter/1} where only used with one Variable, so that the state is first retracted and then newly asserted.\\
	Since the list, from which the user may select, is to be changed with \textit{selected(0)}, the central state management was implemented here. The respective state is changed by every call and at the end is set again to the initial state and for the two calls of state transition, the rule \textit{switchState/2} was implemented. What is special about this method is the treatment of the last state, where the \textit{collection/1} is reset, and the treatment in state veggie, where multiple selection is allowed. The more detailed treatment of a multi-selection is defined in the rule \textit{multipleSelection/1}. The \textit{counter/1} predicate is also used by limiting the multiple selection. Finally, \textit{collection/1}, \textit{counter/1} is being reset, while \textit{state/1} is changed to the next (initial) state.\\
	In the \textit{selected/2} rule, the chosen option is added to the \textit{collection/1} predicate. For this purpose, the corresponding list in the knowledge base is determined via \textit{call/2} and then the list is compared with  the current state. If the state is correct, a list of possible options is created,using the \textit{suggested/2} rule which is then used as a reference. If the selected item is in the created list, the chosen option is added to the \textit{collection/1} via \textit{addToSelection/1}. The \textit{suggested/2} rule filters the results based on the previous choice. First a list out of the \textit{collection/1} predicate is created with the \textit{findnsols/4} rule and then checked if a certain choice has been made. If a choice was detected which is connected to a certain track, the options of this track are returned here as the variable \textit{Output}. Otherwise, the entered list is returned without modification. Furthermore the different tracks are needed, as well as the rules for the specific tracks. For the specific tracks, a list of allowed options is stated in the knowledge base. With the previous rules an endless operation is possible. To view all chosen options, the \textit{done/1} rule was implemented. This rule can be used to print the current selected list. First, \textit{collection/1} is printed as a list to the command line via the function \textit{findnsols/4}. \textit{Options\char`_/1} then outputs the elements in several lines. The rule \textit{printhelpnote()} outputs a hint to the help rule. Furthermore the rule \textit{options/1} was implemented, which outputs a list in several points. In each call the head of the list is printed and the \textit{options\char`_/1} rule is called again with the tail of the list. 
	In order to simplify the interaction, the rules \textit{printhelpnote/0} and \textit{helpsubway/0} were also 	implemented, but they only print information as text.
	
	\lstinputlisting[title=Code]{Subway.pl}
	
	\section{Using the program}
	
	To use the program, you have to navigate to the respective directory and execute swi-prolog. Via ['name of the program ']. the program is loaded. The rules \textit{selected/2}, \textit{helpsubway/0}, \textit{options/1} and \textit{done(1)} are intended for use. Due to the architecture, a certain sequence must first be processed. They have to be processed one after the other:
	\begin{enumerate}
		\item Bread (\textit{breads}),
		\item Main (\textit{main}), 
		\item Vegetables (\textit{veggies}), 
		\item Sauces (\textit{sauce}),
		\item Sides (\textit{sides})
	\end{enumerate}
	 can be selected. In between, the state can be queried again and again with \textit{done(1)} and \textit{options/1} can display the possible options for a step. The selection takes place with \textit{selected/2}.
	
	\section{Conclusion}
	With this architecture the program can be easily extended. No new rules have to be implemented, only the states have to be adapted and optionally a multiple selection has to be considered. Further tracks can also be added, which also require only minor changes during implementation.

\end{document}